
\documentclass[a4paper,UKenglish,cleveref, autoref]{lipics-v2019}
%This is a template for producing LIPIcs articles.
%See lipics-manual.pdf for further information.
%for A4 paper format use option "a4paper", for US-letter use option "letterpaper"
%for british hyphenation rules use option "UKenglish", for american hyphenation rules use option "USenglish"
%for section-numbered lemmas etc., use "numberwithinsect"
%for enabling cleveref support, use "cleveref"
%for enabling cleveref support, use "autoref"


%\graphicspath{{./graphics/}}%helpful if your graphic files are in another directory

\bibliographystyle{plainurl}% the mandatory bibstyle

\title{Group X: Project title} %TODO Please add

\titlerunning{Project title}%optional, please use if title is longer than one line

\author{John Q. Public}{<STUDENT NR> }{<EMAIL ADDRESS>}{}{}

\author{Joan R. Public}{<STUDENT NR> }{<EMAIL ADDRESS>}{}{}

\authorrunning{Group X}

\Copyright{John Q. Public and Joan R. Public}%TODO mandatory, please use full first names. LIPIcs license is "CC-BY";  http://creativecommons.org/licenses/by/3.0/

\ccsdesc[100]{\ }
%\ccsdesc[100]{General and reference}%TODO mandatory: Please choose ACM 2012 classifications from https://dl.acm.org/ccs/ccs_flat.cfm

\keywords{\ }%TODO mandatory; please add comma-separated list of keywords

\category{}%optional, e.g. invited paper

\relatedversion{}%optional, e.g. full version hosted on arXiv, HAL, or other respository/website
%\relatedversion{A full version of the paper is available at \url{...}.}

\supplement{}%optional, e.g. related research data, source code, ... hosted on a repository like zenodo, figshare, GitHub, ...

%\funding{(Optional) general funding statement \dots}%optional, to capture a funding statement, which applies to all authors. Please enter author specific funding statements as fifth argument of the \author macro.

%\acknowledgements{I want to thank \dots}%optional

%\nolinenumbers %uncomment to disable line numbering

\hideLIPIcs  %uncomment to remove references to LIPIcs series (logo, DOI, ...), e.g. when preparing a pre-final version to be uploaded to arXiv or another public repository

%Editor-only macros:: begin (do not touch as author)%%%%%%%%%%%%%%%%%%%%%%%%%%%%%%%%%%
\EventEditors{John Q. Open and Joan R. Access}
\EventNoEds{2}
\EventLongTitle{42nd Conference on Very Important Topics (CVIT 2016)}
\EventShortTitle{CVIT 2016}
\EventAcronym{CVIT}
\EventYear{2016}
\EventDate{December 24--27, 2016}
\EventLocation{Little Whinging, United Kingdom}
\EventLogo{}
\SeriesVolume{42}
\ArticleNo{23}
%%%%%%%%%%%%%%%%%%%%%%%%%%%%%%%%%%%%%%%%%%%%%%%%%%%%%%

\begin{document}

\maketitle

%TODO mandatory: add short abstract of the document
\begin{abstract}
Short description of your work
\end{abstract}

\section{Introduction}
\label{sec:intro}

\subparagraph{Problem statement.}
We need to find a simple-to-use template to start a project report in  LIPIcs v2019.

\subparagraph{Results.}
This document provides such a template.

\subparagraph{Related work.}
LIPIcs provides the basic template and is thus relevant. But there are many others: these don't apply because our problem statement is about LIPIcs v2019.

\section{Modelling}
\label{sec:model}

\subsection{Validity}
\label{ssec:validity}

\subsection{Quality}
\label{ssec:quality}

\section{Algorithms}
\label{sec:algorithms}

\section{Experiments}
\label{sec:exp}

\subparagraph{Questions.}

\subparagraph{Data.}

\subparagraph{Results.}

\section{Conclusions}

Up to here, your report should consist of at most 11 pages.

\clearpage
% make sure to create a file called references.bib
\bibliography{references}

\appendix
\section{The devil is in the details}

You may add as much material (proofs, derivations, figures, tables, ...) as you want in the appendix. But make sure your report can be understood without relying on these.

\end{document}
